%!TEX root = ./article.tex

% Abstract
\begin{abstract}
%(objectives)
Generative Design (GD) allows architects to explore design using a programming-based approach.
Current GD environments are based on existing Computer-Aided Design (CAD) applications, such as AutoCAD or Rhinoceros 3D, which, due to their complexity, are slow and fail to give architects the feedback they need to explore GD.
In addition, current GD environments are limited by the fact that they need to be installed and, therefore, are not easily accessible from any computer.
Architects would benefit from a GD Integrated Development Environment (IDE) in the web that is accessible without installation and that is more interactive than existing GD environments.

%(contents)
This thesis proposes a GD IDE based on web technologies.
Its main component is a web page, containing a program editing interface that allows the architect to make programs and view results in 3D.
To make the editing experience more intuitive, it runs programs whenever they are changed, allows numeric literals to be adjusted by clicking and dragging, and highlights the relationship between program and results.
The IDE also includes a secondary application for exporting results to CAD applications installed in the architect's computer.

%(results + conclusions)
With this approach, we were able to implement a GD environment that is accessible from any computer, offers an interactive editing environment, and integrates easily into the architect's workflow.
In addition, in what concerns program running times, it has good performance that can be one order of magnitude faster than current GD IDEs.
\end{abstract}

% Keywords
\begin{keywords}
  Generative Design; Web technologies; Integrated Development Environments; Architecture
\end{keywords}
