%!TEX root = ../dissertation.tex

\chapter{Conclusion}
\label{chapter:conclusion}
The current trend in architecture practice revolves around using programming as a way to automate repetitive tasks and to explore new design possibilities.
These activities are often put under the umbrella of \acrfull{gd}.

Tools used to support this design approach are mostly desktop applications but this is a leftover from a past that, in many other fields, is already being changed so that applications become web-based.
This change has the significant advantage that applications can be much more easily updated and can be used with whatever machine it is available at that moment.
Moreover, with teams becoming more dispersed geographically, the need arises for using a distributed solution in which team members can still collaborate.
Web applications have the advantage of being platform-independent, requiring only a web browser to be used.

Architects need both 3D modeling and programming capabilities to have an environment where they can do \gls{gd}.
To have a better idea of the features necessary for such environment, Chapter~\ref{chapter:background} explored programming environments for various domains, including \gls{gd}, and also 3D modeling applications.
These included desktop applications like Processing, a programming environment for the visual arts, and Rosetta, a programming environment giving architects program portability for both programming language and \gls{cad} application, and also included web applications like IPython, a programming environment for scientific computing with a web page based \gls{ui}, and OpenJSCAD, an online programming environment for solid modeling.

Firstly, we compared these applications according to their application domain and the purpose they give to programming.
Afterwards, we also compared them according to the editing and the programming paradigm, the persistence and the collaboration support and code editing features.

Following the introduction and comparison of those environments, we described the problem we would address and we proposed our solution in Chapter~\ref{chapter:solution}, which is composed of two components:
(1) a web application, used by the architect to create programs;
and (2) a desktop application that, when run, allows the web application to export results of those programs to \gls{cad} applications.

After defining the architecture, we created a test implementation called Luna Moth.
To help the programming task, Luna Moth runs programs when they change, allows literals to be adjusted by clicking and dragging, and highlights the relationship between program and results in both directions.
In addition, to allow for adjusting literals and highlighting the relationship between program and results, Luna Moth analyzes the program's structure to know where literals are and to instrument it for collection of both results and traceability data.

This allows architects to create programs incrementally.
Instead of having to go through the edit-compile-run cycle to see the results of the changes they made, they make the changes and see the effects immediately.
Afterwards, if they have any doubt on how a certain part of the results came to be, they can use the traceability mechanisms, pointing at that part, to see the part of the program that created them.

After implementing the solution, we evaluated it in different ways.
In Chapter~\ref{chapter:evaluation}, we presented some examples of programs that were created using Luna Moth.
Additionally, we tested Luna Moth's performance by measuring the program running and export times and comparing them to running times on Rosetta, OpenJSCAD and Grasshopper.
Taking those measurements into consideration, we concluded that running programs in Luna Moth is faster than running these \glspl{ide}.
In fact, we were surprised by the difference of at least an order of magnitude from Rosetta and Grasshopper, which shows that \gls{cad} applications are indeed too slow to be used in \gls{gd} exploration.
Regarding export times, we saw that exporting from Luna Moth is slower than running in Rosetta.
This was to be expected since, although we run the program in the web application, we have to wait for network communications.
Nonetheless, the ability to export to \gls{cad} applications remains a great advantage for architects since they only need to do it once: when they reached a final solution.
We also measured the effect that keeping track of traceability data has on program running times.
As a result, we concluded that collecting traceability data makes programs around 10--50\% slower, which is a great achievement considering that every function call is being recorded.
Moreover, when we take into account the benefit that showing the program-result relationship brings to the architect and the \gls{gd} program creation process, the slowdown it brings is well worth it.

In conclusion, with Luna Moth, architects are able to explore \gls{gd} without needing to install and update the programming environment and avoiding the installation and use of heavy-weight \gls{cad} applications, which become slow for \gls{gd} exploration.
By avoiding the complexity of \gls{cad} applications, Luna Moth has been able to have much better performance than other \gls{gd} \glspl{ide} and, therefore, have faster feedback.
In addition, since it is possible to export program results to \gls{cad} when needed, Luna Moth does not tie programs to a particular \gls{cad} application.
By doing all of this, it fulfills the goals that were proposed, namely, being \textit{accessible} from any computer, providing an \textit{interactive} programming environment for exploring \gls{gd}, and \textit{integrating} easily with \gls{cad} applications that architects use and, therefore, into their workflow.

Due to the achieved performance, we are convinced that web browsers can be seen as serious candidates for the next generations of design tools.
We are also convinced that architects need tools dedicated to \gls{gd} that avoid the complexity of current \gls{cad} applications to enhance the interactivity of the programming environment.

As a result, we consider Luna Moth as a good step in the direction \gls{gd} \glspl{ide} should take in the future.

The source code for Luna Moth, along with the source code of the example programs, is available on GitHub at \url{https://github.com/pafalium/gd-web-env}.
It requires a modern web browser, like Google Chrome or Mozilla Firefox, to run the web application, requires Racket version 6.4 or above to run the \textit{remote CAD app}, and requires node.js version 5.10.1 or above to be compiled.
It has been tested both on Google Chrome and Mozilla Firefox, running on Windows 7.
Nonetheless, since these web browsers and Racket are also available for Mac OS, Linux and other versions of Windows, the web environment and the \textit{remote CAD app} are also available for these Operating Systems.


\section{Future Work}
Our solution can support the \gls{gd} approach to architecture, however, there are still areas that can be improved.


\subsection{Programming Environment}
There are many opportunities for improvement of the solution as a programming environment.
The following paragraphs describe some of these and speculate on possible solutions.

\paragraph{Writing aids / Code completion}
The environment does syntax highlighting, runs programs on change, has traceability, and helps adjusting literals, but it does not include help to find the available primitives and how to use them.
This can be improved by providing extensive documentation on the primitive library.
However, this is not extensible to functions and variables that the user introduces in his program.
Alternatively, we can improve this by analyzing the current state of programs and using the resulting information to provide code completion.

\paragraph{Illustrated Programming}
There could also be other ways of implementing traceability that would also enable users to document their programs.
We could transform programs into something like IPython's notebooks or the examples given in Illustrated Programming\cite{Leitao2014illustrated}.

\paragraph{Multiple programming languages}
We can also follow Rosetta's lead and start supporting more than one programming language commonly used by architects, so there is one less barrier for those who already know one language and want to avoid learning a new one.
This would also attract people from the communities around the added languages.

\paragraph{Error reporting and debugging}
Debugging is currently dependent on the web browser developer tools, which might be too complex for architects that do not have professional programming experience.
This suggests that a dedicated debugging tool, targeted to the intended users, would significantly improve the debugging experience.

One possible functionality for this debugging tool would be to try to show incomplete results up to the point where the program reached an error.
On top of this, the tool could show and highlight the parameters of the call that produced the error in the 3D view and also highlight the corresponding call expression in the source code.

\paragraph{Timetable traceability}
We can also improve the environment's traceability.
One way it can be done is by implementing the timetables presented in Learnable Programming\cite{victor2012learnable}.
This would worsen the performance of the running process if implemented naively.
It would nonetheless help architects understand how their programs ran.

\paragraph{Camera controls}
There are also some improvements that can be made to the way the user visualizes the results.
Currently, the 3D view only lets the user navigate the 3D space as if moving around a turntable.
It would be good to give him other ways of navigating, like flying or walking through the model.
Furthermore, it would also be important to let him define different viewpoints that he could quickly switch between to get a better grasp of the results.

\paragraph{Improve code navigation}
Depending on the size of a project, it may be easier or harder to find a particular function or variable definition.
If \gls{gd} projects get big enough, it will be advantageous to support navigation.
Navigation support can take inspiration from LightTable's code document and namespace view.
Other alternative would be the ``go to definition'' functionality present in most software development \glspl{ide}.

\paragraph{Adjustment of literals}
An extension to helping adjust literals could also be done.
Instead of just allowing the architect to change the parameters of his programs by adjusting literals in source code, we could provide something like the 3D view handles from Antimony.
Like so, when the environment detected the definition of positions or vectors from literals, it would display distinct handles in the 3D view that the architect could interact with and, consequently, change the respective definitions in the source code.
This would make the adjustment of parameters representing 3D values more intuitive, helping the architect understand more clearly the purpose of those parameters.

\paragraph{Primitives}
Further work must also be done to implement common primitives in architecture, such as, \gls{csg} operators, control-point based surfaces and lines, or sweep and loft operators, to name a few.

\paragraph{User testing}
Finally, it would be important to have more user testing.
By doing it, we can start steering the environment development according to the needs and difficulties of the users and, therefore, make sure the environment remains both useful and easy to use.


\subsection{Cloud-related Improvements}
Apart from opportunities in the programming environment, there are also some opportunities for improvement in areas that require communication between components.
As before, the following paragraphs describe some of these and speculate on possible solutions.

\paragraph{Security}
One area that needs to be improved is security.
Some of the pertinent points where it must be in place are the communication with remote CAD services, that must be private, authenticated and authorized, and the program running process, that must be sandboxed to stop them from changing the environment's web page for malicious purposes.

\paragraph{Persistence and collaboration}
We presented persistence and collaboration as advantages of cloud applications, however, we did not implement them in our solution.
It would be interesting to investigate these features in the future.
Moreover, it would also be of interest to integrate the solution with cloud version control services like GitHub or cloud storage services like Google Drive (as happened to IPython notebooks with the coLaboratory\footnote{\url{http://colaboratory.jupyter.org/welcome/} (last accessed on 10/10/2016)} and Jupyter Drive.\footnote{\url{https://github.com/jupyter/jupyter-drive} (last accessed on 10/10/2016)} projects)

\paragraph{Export performance}
The performance of the export process can also be improved.
Instead of using HTTP requests to communicate with the application running in the user's computer, a WebSockets connection\cite{rfc6455} could be used to avoid the overhead of HTTP.
Alternatively, the environment could send the entirety of the results on a single HTTP request, therefore, decreasing the time spent waiting for data traversing the network and receiving a response.
Another option would be to run the user's program in the \textit{remote CAD app} instead of the web page.


%Intro/related+solution+evaluation descr+solutions details


%Achievements


%%Future work

%- More support for the GD approach (documentation, notebooks)
%- Support common programming languages like Python and Processing
%- Static analysis + code completion
%- Some sort of error reporting
%- Timetable view + user control of what is on the table
%- Better camera controls
%- LightTable code document, code navigation (large enough projects)
%- Manipulation with handles in the 3D (e.g. Antimony, Bret Victor)
%- More primitives


%- Security: Sanitize code, remote CAD service authentication
%-- User authentication
%-- Program sandboxing
%-- remote CAD service authentication, privacy

%- Persistence: multi-user, projects/packages, versions, sharing, dependencies
%- Github integration (or other cloud version control service)
%- Real-time collaboration

%- Running on WebWorker or on server
%- WebSockets for remote CAD service communication
%- Send entire results to remote CAD service instead of individually


%- Do user testing

%- Define guidelines for the programming language and its use



















%There is still a lot of work to be done in online programming environments for architecture, in fact too much work to be done in a master thesis, at least that's what my advisor says.
