%!TEX root = ../dissertation.tex

\chapter{Introduction}
\label{chapter:introduction}

%Motivation
%Goals
%Thesis Outline




%Identify the problem clearly.
%- There is a need for a widely available 3d modeling tool for architects?
%- Current 3d modeling tools limit are limited to a computer?

%Motivation (Benefits of solving the problem?)
%Goals (What we really want to achieve.)
%Contributions (What we have done that can be used by others.) (Separate "Chapter"?)
%Thesis Outline



%State the requirements. What is there a need for?
%- Higher performance, compared to standard modeling tools.
%- Fit into the current workflow of the architect.




%Introduce the context of the work (the currently used tools, etc).
%Smoothly introduce the problem/need for a solution.
%Lastly, clearly state the goals of the work.

%% Bruno Ferreira, Rosetta Revit BIM
%Apareceram os CADs, (evitam redesenhar tudo à mão)
%Apareceram os BIMs, (modelo computacional de arquitetura)
%Ambos têm muito trabalho repetido
%O Generative Design, Procedural modeling apareceram
%Procedural modeling normalmente limitado a um CAD
%Apareceu o Rosetta
%Rosetta só suporta CADs
%Revit(BIM) tem API
%Vamos usá-la para dar suporte para o Revit ao Rosetta
%
%% Uma secção quando muda de assunto.

%% Projecto de tese
%Programação cada vez mais adoptada
%É preciso aprender muitos conceitos e processos e ter muita disciplina
%Os IDEs juntam todas as ferramentas num pacote
%Os IDEs para software industrial são demasiado para iniciantes
%Há IDEs, feitos para certos domínios, mais amigáveis para iniciantes
%Os arquitetos começaram a usar programação para fazer o seu trabalho e também podem usufruir dos benefícios dos IDEs
%Por exemplo, eles usam os IDEs imbutidos em CADs, o Grasshopper, o Processing e o Rosetta
%Estes IDEs são todos instalados, limitando os computadores onde se pode trabalhar
%Pode-se passar IDE para aplicação web
%Tem de suportar gráficos 3D
%No problem, já existem muitas aplicações web com 3D graças ao WebGL.
%Objectivo: Fazer IDE para arquitetura como uma aplicação web
