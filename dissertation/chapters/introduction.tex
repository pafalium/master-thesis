%!TEX root = ../dissertation.tex

\chapter{Introduction}
\label{chapter:introduction}

%Motivation
%Goals
%Thesis Outline

\section{From Paper to Bits}
Through the years, computers have been taking more ground in the field of architecture.
In the beginning, they were only used for creating technical drawings using \gls{cad} software like AutoCAD.
As opposed to drawing by hand, using computers to create these drawings allowed architects to make changes without having to create other copy of the drawing.
Still, most of the finished work, like project documentation and scale models, was left out of the computer.

As computers became powerful enough to display 3D graphics and 3D modeling started to appear, it became possible to model buildings in the computer and get a 3D preview of them.
Having a 3D model of the building and being able to explore it, to see it from many angles, is much more intuitive and makes it easier to think about the building.
What was only possible to do with a physical scale model or a good perspective drawing was now much easier to do.

Up to this time, many of the documentation needed for a building could be created in the computer but there was a problem: the documentation was fractured into many separate pieces - drawings, 3D models, descriptive text, spreadsheets - each created in its own software.
As the project evolves and changes need to be made, all these need to be manually changed and kept consistent.
This will inevitably introduce errors and will be extremely hard to do in very complex projects.

This can be solved by software that automates the process. Commonly known as \gls{bim} software.


\section{Generative Design / Scripting}
As was previously said, architects now have software tools that replaced most physical artifacts with digital ones.
There is, however, still a problem: the architect can still find tasks that are mostly a repetitive process which is, again, prone to errors.
For example, when modeling a building there are a lot of repetitive tasks that are not trivial to accomplish using the tools that a 3D modeling software provides.

This is what led to the emergence of programming languages for 3D modeling software.%
\footnote{These are often called scripting languages.}
With a programming language, the architect is able to translate the concepts of his problem into a program that can still be run and produce the 3D model he needs in the \gls{cad}.
On top of that, after writing the program the process of generating the model is much faster than the manual equivalent.

Having a faster process for building 3D models means that the architect can explore more variations of a design when it is already modeled since it is no longer too expensive to remodel.
He can now use the program, changing it, to explore a broader design space.
This is what is called \gls{gd}.


\section{Generative Design IDEs}
To create \gls{gd} programs, the architect needs to use a programming language and an \gls{ide}.

The programming language defines what is a program, how it works and gives the architect a conceptual model to think about programs.
For example, it may define a program as a sequence of operations that need to be performed and it may also define primitive operations like creating a box or moving an object.

The \gls{ide} provides tools - editors, compilers, debuggers, among others - that let the architect create programs.
The most basic \gls{ide} may have a text editor, where the architect types a textual representation of the program, and an interpreter, that will interpret the program and perform the required operations.

There can be a variety of \glspl{ide} where programs of a specific programming language can be made and each \gls{ide} may have tools to create programs in several programming languages.

The programming language and the IDE that the architect uses greatly influences the type of programs that he can create.
Some IDEs make it easy to create simple programs really fast but that will not be easy to grow, while others, although harder to use, allow for complex programs.

In some cases, the programming language is inseparable from the IDE - like the case of Grasshopper 3D and Dynamo - while in other cases it does not meter which IDE is used.

Visual IDEs and programming languages are used by most architects that are new to programming.
However, these programming languages do not scale well, i.e. for bigger programs it becomes harder and harder to understand or modify them.

It is important to remember that architects do not have the same programming skills as software developers.
It is also true that architects usually have less programming experience and, therefore, often do not know how to structure their programs, leading to less scalability.


\section{Disadvantages of Desktop Software}
It is common practice in any creative field to carry a notebook or a sketchbook for the sudden moments of inspiration that appear through the day.
This way people make sure that they do not forget any of their ideas.

An aspect common to software tools that architects use is that they need to be installed.
Like so, they are not readily available on every computer.
If architects want to be able to work with their software tools, they have to carry the computer around.
This is why laptop computers are used.
They can run the same applications as desktop computers and they are also portable.
People can carry them around and work anywhere.

Carrying a computer solves the problem of having the tools of trade at hand but what happens when the architect needs to work with a team?
There are cloud-based services that he can use to do it.
He can use services like Skype\footnote{www.skype.com} when speaking with other people or presenting his work, and he use file sharing services to build \gls{cad} models collaboratively.
Still, this is a rudimentary method.
It does not allow more than one person to edit a file simultaneously, it does not have a system to track versions of files and it also only works on desktop-like computers.

This is not a problem specific to architects, in fact, anyone working remotely faces this problem and it has already been solved in other fields that use \gls{cad}.
There are already working cloud-based services that aim to enable this kind of collaboration in other fields.
One example of such services is OnShape\footnote{www.onshape.org}, that addresses the problem for product design / engineering.
To do this, OnShape has real-time collaboration, projects are stored in the cloud, has git-like version control, works both on desktop and on mobile.


\section{Goals}
In this thesis we propose that, with a cloud-based application, architects will be able to work on \gls{gd} projects even while they are outside of their workplace and thus allow them to have more artistic freedom.



%Identify the problem clearly.
%- There is a need for a widely available 3d modeling tool for architects?
%- Current 3d modeling tools limit are limited to a computer?

%Thesis
%- A web application is the natural step to architecture software?
%--- It centralizes the information on the Internet making it more accessible and affordable.
%---

%Motivation (Benefits of solving the problem?)
%Goals (What we really want to achieve.)
%Contributions (What we have done that can be used by others.) (Separate "Chapter"?)
%Thesis Outline



%State the requirements. What is there a need for?
%- Higher performance, compared to standard modeling tools.
%- Fit into the current workflow of the architect.




%Introduce the context of the work (the currently used tools, etc).
%Smoothly introduce the problem/need for a solution.
%Lastly, clearly state the goals of the work.

%% Bruno Ferreira, Rosetta Revit BIM
%Apareceram os CADs, (evitam redesenhar tudo à mão)
%Apareceram os BIMs, (modelo computacional de arquitetura)
%Ambos têm muito trabalho repetido
%O Generative Design, Procedural modeling apareceram
%Procedural modeling normalmente limitado a um CAD
%Apareceu o Rosetta
%Rosetta só suporta CADs
%Revit(BIM) tem API
%Vamos usá-la para dar suporte para o Revit ao Rosetta
%
%% Uma secção quando muda de assunto.

%% Projecto de tese
%Programação cada vez mais adoptada
%É preciso aprender muitos conceitos e processos e ter muita disciplina
%Os IDEs juntam todas as ferramentas num pacote
%Os IDEs para software industrial são demasiado para iniciantes
%Há IDEs, feitos para certos domínios, mais amigáveis para iniciantes
%Os arquitetos começaram a usar programação para fazer o seu trabalho e também podem usufruir dos benefícios dos IDEs
%Por exemplo, eles usam os IDEs imbutidos em CADs, o Grasshopper, o Processing e o Rosetta
%Estes IDEs são todos instalados, limitando os computadores onde se pode trabalhar
%Pode-se passar IDE para aplicação web
%Tem de suportar gráficos 3D
%No problem, já existem muitas aplicações web com 3D graças ao WebGL.
%Objectivo: Fazer IDE para arquitetura como uma aplicação web
