%!TEX root = ../dissertation.tex

\begin{otherlanguage}{portuguese}
\begin{abstract}
\abstractPortuguesePageNumber


Desenho Generative (DG) permite aos arquitetos a exploração de novos tipos de design usando uma abordagem baseada na programação.
Os ambientes de DG atuais baseiam-se nas aplicações de Computer-Aided Design (CAD) existentes, como o AutoCAD e o Rhinoceros 3D, que, devido à sua complexidade, são lentos e não dão o feedback necessário para que os arquitetos explorem DG.
Além disso, os ambientes de DG atuais estão limitados pelo facto de terem de ser instalados e, consequentemente, não estão acessíveis em qualquer computador.
Ter um Ambiente de Desenvolvimento Integrado (IDE em inglês) para DG na web, acessível em qualquer computador sem instalação e pelo menos tão interactivo como os IDEs para DG existentes, seria vantajoso para os arquitetos.

Esta tese propõe um IDE para DG baseado em tecnologias web.
Como componente principal está uma página web, que contem uma interface para edição de programas que por sua vez permite ao arquiteto criar programas e ver os resultados em 3D.
Para tornar a experiência de edição mais intuitiva, o IDE também reexecuta os programas assim que são mudados, permite que literais numéricos sejam ajustados clicando e arrastando, e realça a relação entre programa e resultados.
O IDE também inclui uma aplicação secundária para permitir a exportação de resultados para aplicações CAD instaladas no computador do arquiteto.

Com esta abordagem, conseguimos implementar um ambiente para DG que está acessível a partir de qualquer computador, que oferece uma interface de edição interactiva, e que se integra facilmente no fluxo de trabalho do arquiteto.
Ainda, no que diz respeito a tempos de execução de programas, o ambiente tem um bom desempenho que consegue ser uma ordem de grandeza mais rápido que os IDEs para DG atuais.

% Keywords
\begin{flushleft}

\palavrasChave{Desenho Generativo, Tecnologias Web, Ambientes de Desenvolvimento Integrado, Arquitetura}

\end{flushleft}

\end{abstract}
\end{otherlanguage}
