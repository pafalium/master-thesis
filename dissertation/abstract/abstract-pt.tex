%!TEX root = ../dissertation.tex

\begin{otherlanguage}{portuguese}
\begin{abstract}
\abstractPortuguesePageNumber


Desenho Generativo (DG) permite aos arquitetos criar designs usando uma abordagem baseada na programação.
Os ambientes de DG atuais baseiam-se nas aplicações de \foreign{Computer-Aided Design} (CAD) existentes, como o \foreign{AutoCAD} e o \foreign{Rhinoceros 3D}, que, devido à sua complexidade, são lentos e não dão o \foreign{feedback} necessário para que os arquitetos explorem DG.
Além disso, os ambientes de DG atuais estão limitados pelo facto de terem de ser instalados e, consequentemente, não estarem acessíveis a partir de qualquer computador.
Seria vantajoso para os arquitetos ter um Ambiente de Desenvolvimento Integrado (IDE em inglês) para DG na \foreign{web}, acessível em qualquer computador sem instalação e sendo mais interactivo que os ambientes para DG existentes.

Esta tese propõe um IDE para DG baseado em tecnologias \foreign{web}.
A componente principal é uma página \foreign{web} que contem uma interface para edição de programas que, por sua vez, permite ao arquiteto criar programas e ver os resultados em 3D.
Para tornar a experiência de edição mais intuitiva, o IDE também reexecuta os programas assim que são modificados, permite que os literais numéricos sejam ajustados clicando e arrastando, e realça a relação entre o programa e os resultados.
O IDE também inclui uma aplicação secundária para permitir a exportação de resultados para aplicações CAD instaladas no computador do arquiteto.

Com esta abordagem, conseguimos implementar um ambiente para DG que está acessível a partir de qualquer computador o qual não só oferece uma interface de edição interactiva, como também se integra facilmente no fluxo de trabalho do arquiteto.
Ainda, no que diz respeito a tempos de execução de programas, o ambiente tem um bom desempenho que consegue ser uma ordem de grandeza mais rápido que os IDEs para DG atuais.

% Keywords
\begin{flushleft}

\palavrasChave{Desenho Generativo, Tecnologias Web, Ambientes de Desenvolvimento Integrado, Arquitetura}

\end{flushleft}

\end{abstract}
\end{otherlanguage}
